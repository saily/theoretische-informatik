% Daniel Widerin
\documentclass[10pt,oneside,a4paper]{scrartcl}
\usepackage[ngerman]{babel}
\usepackage[T1]{fontenc}
\usepackage[utf8]{inputenc}
\usepackage[fleqn]{amsmath}
\usepackage{amssymb}
\usepackage{lmodern}

\usepackage{anysize}
\marginsize{2cm}{1cm}{1cm}{1cm}

\usepackage{setspace}
\onehalfspacing

\usepackage{fancyhdr}
\pagestyle{fancy}

\newcommand{\dotminus}{\stackrel{\cdot}{\relbar}}

%\fancyhf{}
%\fancyhead[L]{Formelsammlung ``Theoretische Informatik''}
%\fancyhead[C]{\today}
%\fancyfoot[R]{\thepage}

\begin{document}

\part{Theoretische Informatik A}

\section{Mengen}

    \begin{align}
    &\text{cf}_A :\mathbb{N} \longrightarrow \mathbb{N}
        && \text{rekursive (entscheidbare) Funktion, wenn berechenbar}\\
        &&&: \text{wird gesprochen: $\text{cf}_A$ total von $A$ nach $B$}
        \nonumber\\
    &\text{df}_A :\subseteq \mathbb{N} \longrightarrow \mathbb{N}
        && \text{rekursiv aufzählbare Funktion, wenn berechenbar}\\
        &&&:\subseteq \text{wird gesprochen: $\text{df}_A$ partiell von $A$
        nach $B$}\nonumber\\
    &\perp && \text{Undefiniert}
    \end{align}

\section{Flussdiagramme}

    \begin{align}
    &F   && \text{Flussdiagramm}\nonumber\\
            &&&F = (Q, D, \sigma, q_0)\\
    &KON && \text{Menge der Konfigurationen}\nonumber\\
            &&&KON := Q x D\\
    &ES  && \text{Einzelschrittfunktion}\nonumber\\
            &&&ES :\subseteq KON \longrightarrow KON\\
    &SZ  && \text{Schrittzahlfunktion}\nonumber\\
            &&&SZ :\subseteq KON \longrightarrow \mathbb{N}\\
    &GS  && \text{Gesamtfunktion}\nonumber\\
            &&&GS :\subseteq KON \longrightarrow KON\\
    &f   && \text{Funktion die das Flussdiagramm berechnet}\nonumber\\
            &&&f_F :\subseteq D \longrightarrow D\\
    &\vdash^{*}_{F}
         && \text{Übergangsrelation}\\
            &&&\kappa_{1} \text{ geht über in }\kappa_{2}\nonumber\\
            &&&\kappa_{1}, \kappa_{2} \in KON\nonumber
    \end{align}

\section{Maschinen}

    \begin{align}
    &M   && \text{Maschine}\nonumber\\
            &&&M = (F, X, Y, EC, AC)\\
    &X   && \text{Eingabemenge}\nonumber\\
    &Y   && \text{Ausgabemenge}\nonumber\\
    &EC  && \text{Eingabekodierung}\nonumber\\
            &&&EC : X \longrightarrow D\\
    &AC  && \text{Ausgabekodierung}\nonumber\\
            &&&AC : D \longrightarrow Y\\
    &f_M && \text{Funktion die die Maschine berechnet}\nonumber\\
            &&&f_M := X \longrightarrow Y\\
            &&&f_M := AC \circ f_F \circ EC
    \end{align}

\section{Berechenbare Zahlenfunktionen}

    \begin{align}
    &\tilde{0} : \mathbb{N}^0\longrightarrow \mathbb{N}, \tilde{0}() := 0
        && \text{nullstellige Nullfunktion}\\
    &Z : \mathbb{N}\longrightarrow \mathbb{N}, Z(x) := 0
        && \text{einstellige Nullfunktion}\\
    &S : \mathbb{N}\longrightarrow \mathbb{N}, S(x) := x+1
        && \text{Nachfolgerfunktion (Successor)}\\
    &V : \mathbb{N}\longrightarrow \mathbb{N}, V(x) := x \dotminus 1
        && \text{Vorgängerfunktion}\\
    &pr_i^{(k)} : \mathbb{N}^k\longrightarrow \mathbb{N},
        pr_i^{(k)}(x_1, \ldots, x_k) := x_i \text{ für $1\leq i\leq k$}
        && \text{Projektion}\\
    &f(x_1, \ldots, x_n) := k \text{ für } n, k \in \mathbb{N}
        && \text{konstante Funktion}\\
    &f(x, y) := x+y && \text{Summe}\\
    &f(x, y) := x \dotminus y && \text{arithmetische Differenz}\\
    &f(x, y) := x \cdot y && \text{Produkt}\\
    &q,r \subseteq\mathbb{N}^2\longrightarrow \mathbb{N}
        \text{ mit } x = q(x, y)\cdot q+r(x, y)\nonumber\\
        &\hspace{5mm}\text{ und } r(x, y) < y \text{ für } y \neq 0\nonumber\\
        &\hspace{5mm}\text{ und } q(x, y) = r(x, y) = div \text{ für } y = 0
        && \text{Quotient}\\
    &f(x, y) := x^y \text{ mit } 0^0 := 1 && \text{Exponentiation}\\
    &f(x) := \left\{
        \begin{array}{ll}
        \sqrt{x} & \text{falls $x$ Quadratzahl}\\
        div & \text{sonst}
        \end{array}\right\\
    &max(x, y)\\
    &min(x, y)\\
    &f(x) := \left\{
        \begin{array}{ll}
        1 & \text{falls $x$ Primzahl}\\
        0 & \text{sonst}
        \end{array}\right\\
    &f(x) := \text{ die $x$-te Primzahl}\\
    &f(x, y) := \left\{
        \begin{array}{ll}
        1 & \text{falls }x<y\\
        0 & \text{sonst}
        \end{array}\right
        && \text{Entsprechend für $\leq, =, \geq, >, \neq$.}\\
    &f(x) := \left\{\begin{array}{ll}
        0 & \text{falls } x\in A\\
        1 & \text{sonst}
        \end{array}\right
        && \text{für }A\subseteq\mathbb{N}\text{ endlich. }\\
    &f(x) := \left\{\begin{array}{ll}
        \text{die kleinste Zahl $y$ mit $g(y) = x$} &
        \text{falls }x\in \text{Bild}(g)\\
        div & \text{sonst}
        \end{array}\right
        && \text{für }g:\mathbb{N}\longrightarrow \mathbb{N}
           \text{ berechenbar.}\\
    &f(x) := max\left\{y | g(y) \leq x\right\}
        && \text{für }g:\mathbb{N}\longrightarrow \mathbb{N}
           \text{ berechenbar,}\nonumber\\
        &&& \text{und es gelte $(\forall y) g(y) < g(y+1)$}\\
    &f(x, y) := \left\{
        \begin{array}{ll}
        ggT(x, y) & \text{falls }x, y > 0\\
        0 & \text{sonst}
        \end{array}\right
    \end{align}

    \subsection{Cantorsche Paarungsfunktion}

    \begin{align}
    &\pi : \mathbb{N}^2 \longrightarrow \mathbb{N}
            &&\text{Cantorsche Tuplefunktion ist bijektiv}\nonumber\\
            &&&\pi(x, y) = \sum^{x+y}_{i = 0} i+y =\frac{1}{2}(x+y)(x+y+1)+y\\
            &&&\text{für alle $x, y \in \mathbb{N}$.}\nonumber\\
    &\pi^{(k)} : \mathbb{N}^k \longrightarrow \mathbb{N}\nonumber &&\\
            &&&\pi^{(1)}(x) := x\nonumber\\
            &&&\pi^{(k+1)}(x_1, \ldots, x_{k+1}) :=
               \pi (\pi^{(k)}(x_1, \ldots, x_k), x_{k+1})\\
            &&&\text{für alle $k, x, x_1, \ldots, x_{k+1} \in \mathbb{N}$ mit}
               \text{$k \geq 1$.}\nonumber\\
    \end{align}

    %\subsubsection{Umkehrfunktion}

    \begin{align}
    &\langle x_1, \ldots, x_k\rangle := \pi^{(k)}(x_1, \ldots, x_k)\\
    &\pi^{(k)} && \text{ist bijektiv und berechenbar}\\
    &\pi_i^{(k)} := pr_i^{(k)}(\pi^{(k)})^{-1}
        && \text{ist berechenbar für jedes $i$ mit $1 \leq i \leq k$.}\\
    &\pi_1 := \pi_1^{(2)}\text{, }\pi_2 := \pi_2^{(2)}
        && \text{Kurzschreibweise}\nonumber\\
    \nonumber\\
    &f(\omega) := \sum_{i=0}^{\omega} i = \frac{1}{2} \omega (\omega+1)\\
    &q(z) := max\{v|f(v) \leq z\}
    \end{align}

\subsection{Eigenschaften berechenbarer Zahlenfunktionen}

    \begin{align}
    &P^{(k)} &&:= \left\{f :\subseteq\mathbb{N}^k\longrightarrow
        \mathbb{N} | f \text{ berechenbar}\right\}\\
    &R^{(k)} &&:= \left\{f :\mathbb{N}^k\longrightarrow
        \mathbb{N} | f \text{ berechenbar}\right\}\\
    &P &&:= \bigcup_{k\in\mathbb{N}} P^{(k)}
        && \text{berechenbar bzw. partiell rekursiv}\\
    &R &&:= \bigcup_{k\in\mathbb{N}} R^{(k)}
        && \text{total berechenbar bzw. total rekursiv}\\
    \end{align}

    \begin{align}
    &Sub(g, h_1, \ldots, h_m) :\subseteq\mathbb{N}^k\longrightarrow \mathbb{N}
        &&Sub(g, h_1, \ldots, h_m)(\bar{x}) := g(h_1(\bar{x}), \ldots,
            h_m(\bar{x}))\\
    &g:\mathbb{N}^{k}\longrightarrow \mathbb{N} && f(\bar{x}, 0) = g(\bar{x})\\
    &h:\mathbb{N}^{k+2}\longrightarrow \mathbb{N} && f(\bar{x}, y+1) = h(
        \bar{x}, y, f(\bar{x}, y))\\
    &Prk(g,h) f : \mathbb{N}^{k+1} \longrightarrow \mathbb{N}
        &&\tilde{\mu}(h)(\bar{x}) = \mu y \left[ h(\bar{x}, y) = 0 \right]
    \end{align}

\section{Primitiv rekursive Funktionen}

    \begin{align}
    &Gr := \left\{\tilde{0}, Z, S\right\} \cup
           \left\{pr_i^{(k)}|i,k\in\mathbb{N}, 1\leq i\leq k\right\}
        && \text{Menge der primitiv-rekursiven Grundfunktionen (PRK)}
    \end{align}

    \begin{align}
    &c_n^{(k)} : \mathbb{N}\longrightarrow \mathbb{N},
        c_n^{(k)}(x_1,\ldots,x_k) := n \text{ für alle }k,n\in\mathbb{N}
        && \text{konstante Funktionen}\\
    &s : \mathbb{N}^2\longrightarrow \mathbb{N}, s(x, y) := x+y
        && \text{Summe}\\
    &V : \mathbb{N}\longrightarrow \mathbb{N}, V(x) := x \dotminus 1
        && \text{Vorgängerfunktion}\\
    &d : \mathbb{N}^2\longrightarrow \mathbb{N}, d(x, y) := x \dotminus y
        && \text{arithmetische Differenz}\\
    &m : \mathbb{N}^2\longrightarrow \mathbb{N}, m(x, y) := x \cdot y
        && \text{Multiplikation}
    \end{align}

\section{Turingmaschinen}

\section{Bandmaschinen}

    Bandmaschinen sind Turingmaschinen mit nur einem einzigen Band. Durch die
    virtuelle Kopfposition und ein neues Arbeitsalphabeth $\Gamma'$ kann man
    zeigen dass die Bandmaschine $f'_M$ gleich der Turingmaschine $f_M$ ist und
    somit gilt $f_M = f'_M$.

\section{Rekursive Mengen}

\subsection{Reduzierbarkeit}

    \begin{align}
    &A \leq B \Longleftrightarrow \exists f \in R^{(1)}.A=f^{-1}[B]
        && \text{$A, B \subseteq\mathbb{N}$. $A$ ist reduzierbar auf $B$.}\\
    &A \leq B \Longleftrightarrow \text{cf}_A\leq cf_B
    \end{align}

\pagebreak

\part{Theoretische Informatik B}

\section{Maschinenmodelle und Komplexitätsklassen}

    \begin{align}
    &L_M := \left\{x\in\Sigma^*|f_M(x)=\epsilon\right\}
        && \text{die von $M$ erkannte, entschiedene Sprache}\\
    &t_M :\subseteq\Sigma^*\longrightarrow\mathbb{N}
        && \text{Rechenzeit, Zeitkomplexität von $M$}\\
    &s_M := \sum_{i=0}^{k+1} (lg(u_i) + lg(v_i) + 1)
        &&\text{Bandbedarf, Bandkomplexität oder Speicherkomplexität von $M$}\\
    &\tilde{t_M}(n) := max\left\{t_M(x) | x\in\Sigma^n\right\}
        &&\text{und }\tilde{t_M}:\subseteq\mathbb{N}\longrightarrow\mathbb{N}\\
    &\tilde{s_M}(n) := max\left\{s_M(x) | x\in\Sigma^n\right\}
        &&\text{und }\tilde{s_M}:\subseteq\mathbb{N}\longrightarrow\mathbb{N}\\
    &\log{n} := \left\{\begin{array}{ll}
        1 & \text{falls }n=0\\
        \lfloor\log_2{n}\rfloor + 1 & \text{sonst}
    \end{array}\right
        && \text{Zahl der Ziffern der Dualnotation von $n$}
    \end{align}

\section{O-Notation}

    Sei $g : \mathbb{N}\longrightarrow\mathbb{N}$ eine (totale) Funktion.
    \begin{align}
    &O(g) := \left\{f | f : \mathbb{N}\longrightarrow\mathbb{N},
                    \exists c \in \mathbb{N} .
                    \forall n \in \mathbb{N} .
                    f(n) \leq c \cdot g(n) + c
             \right\}
        && \text{sprich: ``groß Oh von g''}\\
    &o(g) := \left\{f | f : \mathbb{N}\longrightarrow\mathbb{N},
                    \lim_{n\rightarrow\infty} \frac{f(n) + 1}{g(n) + 1} = 0
             \right\}
        && \text{sprich: ``klein oh von g''}
    \end{align}

\section{Komplexitätsklassen}
    Sei $f : \mathbb{N}\longrightarrow\mathbb{N}$ eine (totale) Funktion,
    $\Sigma$ ein Alphabet.
    \begin{align}
    &\textsf{ZEIT}_{\Sigma}(f) := \left\{
        L_M | M \text{ Turingmaschine über }
        \Sigma, \tilde{t}_M \in O(f)\right\}\\
    &\textsf{FZEIT}_{\Sigma}(f) := \left\{
        L_M | M \text{ Turingmaschine über }
        \Sigma, \tilde{t}_M \in O(f)\right\}\\
    &\textsf{BAND}_{\Sigma}(f) := \left\{
        L_M | M \text{ Turingmaschine über }
        \Sigma, \tilde{s}_M \in O(f)\right\}\\
    &\textsf{FBAND}_{\Sigma}(f) := \left\{
        L_M | M \text{ Turingmaschine über }
        \Sigma, \tilde{s}_M \in O(f)\right\}\\
    &\textsf{ZEIT}(f) := \bigcup_{\Sigma}{\textsf{ZEIT}_{\Sigma}(f)}
    \end{align}

\end{document}
